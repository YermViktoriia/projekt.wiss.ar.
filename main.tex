\documentclass{article}
\usepackage{graphicx}

\title{Analyse des Titanic-Datensatzes}
\author{Viktoriia Yermolaieva}
\date{Februar 2026}

\begin{document}

\maketitle

\section{Auswahl der Funktionen}

Für die Analyse des Titanic-Datensatzes wurden gezielt Funktionen entwickelt, 
die eine sinnvolle deskriptive Auswertung der Daten ermöglichen. Ziel war es, 
sowohl metrische als auch kategoriale Variablen systematisch untersuchen zu können.

Die Funktion \textbf{desc\_numeric()} wurde implementiert, um zentrale deskriptive 
Kennzahlen für metrische Variablen wie Alter oder Ticketpreis zu berechnen.

Die Funktion \textbf{desc\_categorical()} dient zur Analyse kategorialer Variablen 
wie Geschlecht oder Einschiffungshafen.

Die Funktion \textbf{biv\_cat\_cat()} wurde erstellt, um Zusammenhänge zwischen zwei 
kategorialen Variablen zu untersuchen.

Die Funktion \textbf{biv\_metric\_dicho()} ermöglicht die Analyse des Zusammenhangs 
zwischen einer metrischen Variable und einer dichotomen Variable.

Zusätzlich wurde die Funktion \textbf{plot\_multi\_cat()} implementiert, um mehrere 
kategoriale Variablen gleichzeitig visuell darzustellen.

\end{document}
